\chapter{Background Theory}

% \newenvironment{conditions}
%   {\par\vspace{\abovedisplayskip}\noindent\begin{tabular}{>{$}l<{$} @{${}={}$} l}}
%   {\end{tabular}\par\vspace{\belowdisplayskip}}
  
\newenvironment{conditions}[1][where:]
  {#1 \begin{tabular}[t]{>{$}l<{$} @{${}={}$} l}}
  {\end{tabular}\\[\belowdisplayskip]}

\label{ch:background}

% \section{Autism Spectrum Disorders (ASD)}

% Nowadays, there does not exist a recognised medical test for autism diagnostic. Cases are examined individually by doctors for classification. Online screening tools such as Q-Chat are currently available to help parents understand if their child is affected or not by autism \cite{screening}. Uses of Machine Learning (to analyse patients EEG readings) and Computer Vision \footnote{to detect, from video recording, behavioural and communication impairments} might provide a useful solution to this problem. 

% Each LSTM module can be considered to be formed by three gates: input ($i_{t}$) , forget ($f_{t}$) and output gate ($o_{t}$). The input gate decides if a piece of information is important enough or not to be remembered (Equation 2.1). The forget gate (Equation 2.2) decides if a piece of information stored is still relevant or not (and therefore has to be deleted). The output gate determines if a particular information has to have or not a weight in the current time step (Equation 2.3).
% \useshortskip
% \begin{align}
% \ i_{t} = \sigma(w_{i}[h_{t-1},X_{t}] + b_{i}) \\
% \ f_{t} = \sigma(w_{f}[h_{t-1},X_{t}] + b_{f}) \\
% \ o_{t} = \sigma(w_{o}[h_{t-1},X_{t}] + b_{o})
% \label{eq:3}
% \end{align}
% \useshortskip
% \begin{conditions}
%  w_{x}  &  neuron gate weight \\
%  h_{t-1}     &  LSTM previous step output \\
%  X_{t}     &  LSTM current input \\   
%  b_{x}      &  gates biases \\
% \end{conditions}
% \useshortskip
% The Sigmoid function ($\sigma$, Equation 2.4) is used to squish the output between any value from zero (making the gate block everything) to one (making the gate pass through everything). A neuron gate weight ($w_{x}$) represents the strength of the connection, while the bias ($b_{x}$) is used shift the activation function to fit best the data.

% \useshortskip
% \begin{align}
% \ \sigma(x) = \frac{1}{1 + e^{-x}}
% \label{eq:3}
% \end{align}
% \useshortskip

% In mathematical terms, convolution ($\circledast$) is an operation between two functions to create a third one, which represents to what extent a function can be modified by another (Equation 2.5).
% \begin{align}
% \ (f \circledast g)' = \int_{\infty}^{\infty} f(\tau)g'(t-\tau)d\tau
% \label{eq:3}
% \end{align}
% \useshortskip

% {
% \begin{table}[h!]
% \centering
% \begin{tabular}{l|l|c|c|c}
% \multicolumn{2}{c}{}&\multicolumn{2}{c}{Model Predictions}&\\
% \cline{3-4}
% \multicolumn{2}{c|}{}&Negative (0)&Positive (1)&\multicolumn{1}{c}{Total}\\
% \cline{2-4}
% \multirow{}{}{True Outputs}& Negative (0) & $TN$ & $FP$ & $TN+FP$\\
% \cline{2-4}
% & Positive (1) & $FN$ & $TP$ & $FN+TP$\\
% \cline{2-4}
% \multicolumn{1}{c}{} & \multicolumn{1}{c}{Total} & \multicolumn{1}{c}{$TN+FN$} & \multicolumn{    1}{c}{$FP+TP$} & \multicolumn{1}{c}{$N$}\\
% \end{tabular}
% \caption{Confusion Matrix}
% \label{table:1}
% \end{table}
% }

% Using the Confusion Matrix, the model Accuracy can be also calculated as:
% \begin{align}
% \ Accuracy (ACC) = \dfrac{TP + TN}{TP + TN + FP + FN}\times100\% 
% \end{align}

% From the Confusion Matrix it is then possible to calculate a model Sensitivity and Specificity. In a medical context, sensitivity quantifies the model's ability to determine patients effected by a certain medical condition. Specificity instead measures the model ability to correctly identify patients not effected by this condition.

% \begin{align}
% \ Sensitivity (TPR) = \dfrac{TP}{TP + FN}\times100\% \label{eq:1} \\
% \ Specificity (TNR)  = \dfrac{TN}{TN + FP}\times100\%
% \end{align}

% The AUC (Area Under The Curve) - ROC (Receiver Operating Characteristics) curve evaluates a model's ability to correctly discriminate between the different classes using different thresholds. 

% Plotting the False Positive Rate (Specificity) against the  True Positive Rate (1 - Sensitivity), the ROC Curve can be calculated (Figure 2.10).