\begin{appendices}
%\appendix
\renewcommand\thefigure{\thesection.\arabic{figure}} 
\setcounter{figure}{0}

%\section*{Appendices}
%\addcontentsline{toc}{chapter}{Appendices}
\renewcommand{\thesection}{\Alph{section}}

\definecolor{codegreen}{rgb}{0,0.6,0}
\definecolor{codegray}{rgb}{0.5,0.5,0.5}
\definecolor{codepurple}{rgb}{0.58,0,0.82}
\definecolor{backcolour}{rgb}{0.95,0.95,0.92}
\definecolor{deepred}{rgb}{0.6,0,0}
 
\lstdefinestyle{mystyle}{
    backgroundcolor=\color{backcolour},   
    commentstyle=\color{codegreen},
    keywordstyle=\color{blue},
    numberstyle=\tiny\color{codegray},
    stringstyle=\color{codepurple},
    basicstyle=\footnotesize,
    breakatwhitespace=false,         
    breaklines=true,                 
    captionpos=b,                    
    keepspaces=true,                 
    numbers=left,                    
    numbersep=5pt,                  
    showspaces=false,                
    showstringspaces=false,
    showtabs=false,                  
    tabsize=2
}
 
\lstset{style=mystyle}


\includepdf[pages=-,pagecommand={\section{Project Management}\label{man}\null\vfill\captionof{figure}{Planned Gantt Chart}},noautoscale=true,offset=0 -10, scale=0.70]{images/gann.pdf}

% \includepdf[pages=-,pagecommand={\null\vfill\captionof{figure}{Actual Gantt Chart}},noautoscale=true,offset=0 -10, scale=0.65]{images/ganttreal.pdf}

\includepdf[pages=-,pagecommand={\null\vfill\captionof{figure}{Risk Assesment Matrix}},noautoscale=true,offset=0 -10, scale=0.7]{images/Management.pdf}

\section{Logistic/Exponential Curve Fitting}
\label{exp_fit}

For a logistic curve at the turning point: 

\useshortskip
\begin{align}
\ Slope = Growth Factor/2 \Rightarrow\quad Doubling Time (DT) = \dfrac{ln(2)}{Growth Factor/2}
\end{align}
\useshortskip

Instead, for an exponential curve:

\useshortskip
\begin{align}
\ Slope = Growth Factor \Rightarrow\quad Doubling Time (DT) = \dfrac{ln(2)}{Growth Factor}
\end{align}
\useshortskip

A worked out example, with the results from the top three countries with the most number of Coronavirus Cases as of the end of June 2020, is available below. From this example, we can easily see how well our data resembles a logistic/exponential curve (using the $R^{2}$ score to quantify the mismatch) and what's the predicted time for the number of cases to double given the current trends.

\begin{figure}[ht!]%
    \centering
    \includegraphics[width=1\linewidth]{latex/images/fitting.pdf}
    % \includegraphics[width=15.5cm]{latex/images/fitting.PNG}%
    \caption{Curve Fitting}
\end{figure}

\clearpage

\section{Compartmental Models Causal Diagrams}
\label{causal_comp}


\clearpage

\section{Population Modelling Pseudo-code}
\label{code_alg}

\begin{algorithm}
\caption{Population Modelling Pseudo-code Outline}
\label{alg1}
% \vspace*{-.5cm}
% \begin{multicols}{2}
\begin{algorithmic}[1]
  \STATE $E \Leftarrow Contact\:Radius$
  \STATE $\overline{p} \Leftarrow Unlikeliness\:of\:Spread$
  \STATE $p\_died \Leftarrow Death\:Probability\:dependent\:on\:age$
  \FOR{$day\:in\:simulation\:days$}
  \FOR{$individual\:in\:population$}
      \STATE$Record\:individual\:status$
      \IF{$Infected$}
      \IF{$Draw\:with\:probability\:(p\_died\times age == 1)$}
        \STATE $individual\:status \Leftarrow Dead$
      \ELSE
        \IF{$(Rehabilitation\:days == 14)$}
          \STATE $individual\:status \Leftarrow Recovered$
        \ENDIF
        \STATE $Rehabilitation\:days\:+= 1$
      \ENDIF
      \ELSIF{$Susceptible$}
        \STATE $close\_people = 0$
        \FOR{$friend\:in\:community$}
        \IF{$(Friend==Infected)\:and\:(Euclid\:Dist<E)$}
        \STATE $close\_people\:+= 1$
        \ENDIF
        \ENDFOR
        \IF{$(Draw\:with\:probability\:dependent\:on\:close\_people\:and\:\overline{p} == 1)$}
          \STATE $individual\:status \Leftarrow Infected$
        \ENDIF
      \ENDIF
      \IF{$(Static == False)$}
          \STATE $individual\:X\:and\:Y\:position\:update$
          \IF{$individual\:X\:or\:Y\:position\:out\:of\:boundaries$}
            \STATE $Adjust\:position\:and\:reverse\:movement\:direction$
          \ENDIF
      \ENDIF
  \ENDFOR
\ENDFOR
\end{algorithmic}
% \end{multicols}
% \vspace*{-.4cm}
\end{algorithm}

\clearpage
\includepdf[pages=-,pagecommand=\section{Project Brief}\label{brief},noautoscale=true,offset=0 -70, scale=1.1]{latex/images/Outline.pdf}

\clearpage
\section{Design Archive Guide}
\label{archive}
A tree representation of this Project Design Archive is represented in the figure below. This image has been created through the windows command prompt using the tree command in the designed directory.
% \vspace*{-17mm}
% \setcounter{figure}{0}
% \begin{figure}[ht!]%
%     \centering
%     \includegraphics[page=1,scale=0.88]{images/dirtree.pdf}
%     \vspace*{-52mm}
%     \caption{Design Archive}%
% \end{figure}

\clearpage


\section{Word Count}
\label{count}

The registered word count for this report, starting from Chapter 1 to Chapter 6, was equal to 14,823 words.  This has been measured using Doc Word Counter \cite{count_w}.

\end{appendices}