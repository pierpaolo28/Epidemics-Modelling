\chapter{Conclusion}

\label{ch:planwork}
\setlength\lineskip{0pt}
\vspace*{15pt}

Overall, this project had a positive outcome and all the objectives established in the Project Brief (Appendix \ref{brief}) have been accomplished.

\section{Further Advancements}

In order to take this project further, different aspect could be taken in consideration such as:
\vspace{-0.2cm}
\begin{itemize}
    \item Using NPL in order to find patterns in research publications about research advancements in order to find a cure/vaccine against Coronavirus. In this way, it could be possible to find out if there is any underlying pattern which could be used in order to make progresses (e.g. by combining different approaches).
    \item As more data about Clinical Trials would become available in the following months, it could be possible to make use of it in order to find causal relationships in how different treatments might effect different patients. 
    \item Introduce other causality related techniques such as Knowledge Graphs and Explainable AI. 
    \item Make use of open source causality libraries such as Microsoft DoWhy \cite{dowhy}, Uber CausalML \cite{causalml} and QuantumBlack CausalNex \cite{causalnex}. 
\end{itemize}
\vspace{-0.2cm}
Successful introduction of causality in Machine Learning could lead to a wide adoption of the employed techniques (filling a market gap) and also have important legal consequences. In fact, use of Machine Learning in important decision-making applications is currently under probe by governments and institutions due to the risks involved.

Enabling Machine Learning models to be more easily examinable to gain insights of their decision-making processes could in fact facilitate adoption of this kind of technologies in fields such as Medicine, Surveillance and Hiring.