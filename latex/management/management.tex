\chapter{Project Management}

\label{ch:management}
\setlength\lineskip{0pt}
\vspace*{15pt}

\section{Time Management}
Throughout this project I made use of backup repositories such as:

\begin{itemize}
\itemsep0em
\item \textbf{Dropbox}: to store files related to this project, and to be able to access them from any type of station at any moment.
\item \textbf{GitHub}: to retain a version-control of all the written code, Latex files and create a website page to interactively share code and simulation animations\footnote{Additional information available in Appendix \ref{repo}.}
\item \textbf{Amazon AWS EC2 Instance}: to store the Epidemic Modelling Dashboard Application and create the live version.
\item \textbf{Trello Board}: all the different planned tasks have been recorded and divided on an online Trello board in order to easily plan and keep track of what tasks are left to do and any possible related reference list.
\end{itemize}

In Appendix \ref{man}, there are additionally available a series of project management techniques which have been used in order to best organise and plan this project.

\begin{enumerate}
\itemsep0em
\item A Gantt Chart representing the planned project-schedule.
\item A Gantt Chart summarising the actual project-schedule.
\item A Risk Assessment Matrix summing up all the possible risks related to this study.
\item A Work Breakdown Structure displaying in a tree-like format the main project milestones. 
\end{enumerate}

Finally, in Appendix \ref{archive} there is available the Design Archive Guide, while in Appendix \ref{count} is registered the total Word Count for the project report.

\section{Data Management}
In order to complete this project, use has been made of different freely available data sources in order to create the Epidemic Modelling Dashboard. All the different data sources used, have been acknowledged and referenced as part of this research study. 

\section{Project Challenges}
One of the greatest challenges faced as part of this project, was creating the Agent Based Models outlined in Section \ref{agent_smith}. These models have in fact been created entirely from scratch (not having to follow strictly any mathematical model). Although, taking this approach made possible to build an easily scalable framework able to incorporate different society aspects such as spatial division in communities and economy simulations. 

The different components composing these models have in fact been constructed in order to closely resemble SIR based models, while making it possible to easily incorporate much more complicated dynamics. All the different formulas used in order to design these systems had then to be constructed by hand and through experimentation (e.g. Death probability, Income Update).