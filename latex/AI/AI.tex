\chapter{COVID-19 Analysis}
\setcounter{secnumdepth}{5}
\label{ch:AI}
\setlength\lineskip{0pt}
\vspace*{15pt}

\definecolor{codegreen}{rgb}{0,0.6,0}
\definecolor{codegray}{rgb}{0.5,0.5,0.5}
\definecolor{codepurple}{rgb}{0.58,0,0.82}
\definecolor{backcolour}{rgb}{0.95,0.95,0.92}
\definecolor{deepred}{rgb}{0.6,0,0}
 
\lstdefinestyle{mystyle}{
    backgroundcolor=\color{backcolour},   
    commentstyle=\color{codegreen},
    keywordstyle=\color{blue},
    numberstyle=\tiny\color{codegray},
    stringstyle=\color{codepurple},
    basicstyle=\footnotesize,
    breakatwhitespace=false,         
    breaklines=true,                 
    captionpos=b,                    
    keepspaces=true,                 
    numbers=left,                    
    numbersep=5pt,                  
    showspaces=false,                
    showstringspaces=false,
    showtabs=false,                  
    tabsize=2
}
 
\lstset{style=mystyle}

During Chapter \ref{ch:progress}, we had the opportunity to understand how to potentially model a disease spreading in a community. In this chapter, we are instead going to expand this topic in order to take into account how the spreading of the disease can be forecasted (using real world data) and how comorbodities can be taken into account in the case of COVID-19.

\section{SIR Time Series Estimation}
\label{tool_ref_app}
Building on from the SIR model constructed in Section \ref{sir_sec}, we can make use of it in order to approximate real world data and predict future trends \cite{atom}.

This process can be summarised in the following two steps:

\begin{enumerate}
    \item Estimating $\beta$ and $\gamma$, given the data about the number of cases and recovered in a country. In order to optimise iteratively these two parameters, a modified form of the Root Mean Squared Error (RMSE) equation has been used to take into account of both the number of cases series and the number of recovered individual (Equation \ref{loss_it}). In Equation \ref{loss_it}, there has been additionally added an hyper-parameter ($\alpha$) to decide if to give more weight either to optimising the number of cases or recovered curve fit. This exercise has therefore been designed to be an optimization minimization problem, in which the parameters estimation is improved iteratively by minimising the overall loss. This optimization process is then carried out making use of the Limited-memory BFGS (L-BFGS) algorithm. The L-BFGS algorithm is an approximation\footnote{Limited-memory BFGS, approximating the traditional BFGS algorithm manages to in fact to keep a linear memory consumption.} of the traditional BFGS (Broyden–Fletcher–Goldfarb–Shanno) algorithm which works by estimating the Inverse of the Hessian Matrix in order to move through the search space. The calculated $\beta$ and $\gamma$ are then going to be used as our parameters for the SIR model.
    
    \useshortskip
    \begin{align}
    \ Loss = \alpha \times \sqrt{\dfrac{1}{n} \sum_{t=1}^{n} (I_{t} - ID_{t})^{2}} + (1 - \alpha) \times \sqrt{\dfrac{1}{n} \sum_{t=1}^{n} (R_{t} - RD_{t})^{2}}
    \label{loss_it}
    \end{align}
    \vspace{-0.4cm}
    \begin{conditions}
     $I_{t}, R_{t}$  &  SIR Infected and Recovered timestep \\
     $ID_{t}, RD_{t}$  &  Infected and Recovered from data timestep \\
     $n$  &  Number of timesteps\\
    \end{conditions}
    \vspace{-0.2cm}
    \useshortskip
    \item Solve the SIR model equations by numerically integrating them and providing some initial condition values for the number of susceptible, infected and recovered in the population. The values of the initial conditions can then be calculated by taking into account the population size of the country we are examining and the number of cases registered so far. The integration method used instead to solve the system of differential equations ($4^{th}$ order for 3 dimensions), was the Explicit Runge-Kutta method \cite{kutta}.
\end{enumerate}


In Figure \ref{sir_forecast} are available the prediction results of Italy and Germany as of the $9^{th}$ of July 2020 in order to predict the following 30 days trends. As can been seen from the results, both countries have been fairly well approximated and the number of infected predicted in the simulation have been slightly overestimated. This mismatch although can still look quite realistic in reality because of the limited amount of tests available and presence of asymptomatic patients.

% \vspace{-0.1cm}
\begin{figure}[ht!]%
    \centering
    \includegraphics[width=0.49\linewidth]{latex/images/Italy_preds.pdf}
    \includegraphics[width=0.49\linewidth]{latex/images/Germany_preds.pdf}
    % \includegraphics[width=13cm]{latex/images/vacc.PNG}%
    \vspace{-0.2cm}
    \caption{Italy and Germany SIR Forecasting}
    \label{sir_forecast}
\end{figure}
% \vspace{-0.1cm}

\subsection{ML Forecasting}

Another possible approach which can be taken in order to forecast time series is to use standard Machine Learning and Deep Learning techniques. In this case, the number of infected cases in Germany over time is going to be taken as an example \footnote{Using data up to the $9^{th}$ of July 2020.} to forecast the number of cases in 30 days time. In order to accomplish this task, the Python Darts library \cite{darts} has been used and the following models have been taken into consideration:
\vspace{-0.5cm}
\begin{enumerate}
    \item \textbf{Auto ARIMA (Auto Regressive Integrated Moving Average)}: is a time series method which can be used in order to make predictions as a linear weighted sum of past input data. In the Auto version of ARIMA, the model parameters are automatically inferred through differencing tests and optimised by recording Information Criterion (e.g. Akaike Information Criterion (AIC)) metrics.
    \item \textbf{Exponential Smoothing}: this technique follows the same approach of standard ARIMA models, but the model assigns exponentially decreasing weights for past observations.
    \item \textbf{LSTM (Long-Short-Term-Memory)}: The LSTM is a type of Recurrent Neural Network (RNN) ideated in order to add a memory mechanism suitable to analyse time series (the information is kept in a loop and data is fed in sequentially).
    \item \textbf{T-CNN (Temporal Convolutional Neural Network)}: The T-CNN is a type of Convoluational Neural Network used for time series forecasting. This type of model is composed by a one dimensional convoluational network combined with causal convolutions. In causal convolutions, outputs at a specific timestep are convolved just with elements of the same timestep and of previous layers (in order to add time dependencies).
\end{enumerate}
\vspace{-0.3cm}
The results obtained from this analysis are available in Figure \ref{ml_forecast}. 

In this case, all the models have been used in order to predict a portion of the current series (so that to estimate a fit loss) and predict the next 30 days in the future of the number of cases in Germany.

\vspace{-0.3cm}
\begin{figure}[ht!]%
    \centering
    \includegraphics[width=0.55\linewidth]{latex/images/Germany_darts.pdf}
    % \includegraphics[width=13cm]{latex/images/vacc.PNG}%
    \vspace{-0.2cm}
    \caption{Germany ML Forecasting}
    \label{ml_forecast}
\end{figure}
\vspace{-0.4cm}

In this case, MAPE (Mean Absolute Percentage Error) has been used as our loss function. 
MAPE is in fact one of the most commonly used loss function for regression tasks (Equation \ref{mape}). It's main advantages are interpratibility (we work using percentage terms) and scale-independency. One of the main disadvantages of MAPE is that it can be undefined for actual values close to zero.

\useshortskip
\begin{align}
\ MAPE = \dfrac{1}{n} \sum_{t=1}^{n} \abs{\frac{A_{t}-F_{t}}{A_{t}}} \times 100
\label{mape}
\vspace{-1.8cm}
\end{align}
\vspace{-0.5cm}
\begin{conditions}
 $A_{t}, F_{t}$  &  Actual and forecasted time-step \\
 $n$  &  Number of datapoints\\
\end{conditions}
\vspace{-0.2cm}
\useshortskip

Overall, the LSTM and T-CNN managed to best fit the original data, while both Exponential Smoothing and Auto ARIMA predicted an increase in the number of cases over the following month. 

Complex Deep Learning models are currently able to offer us good performances on a wide variety of tasks but they heavily rely on past data and they are not able to give us any insight about how the system might work behind the scenes. On the other hand, using for example an SIR model in order to make predictions, can allow us to not only to make estimates (like just done using ML based techniques), but also to infer underlying epidemiology parameters such as $\beta$ and $\gamma$ which can in turn give us more information about how the disease is spreading in a community.

\section{Coronavirus comorbidities}
\label{como_app}
Two of the greatest factors which seem to have the greatest impact over the mortality of Coronavirus for different patients, are age and possible pre-existing conditions. In different models implemented in Chapter \ref{ch:progress}, the age factor has been taken into account by varying the mortality likelihood depending on age. In this section\footnote{Making us of the data provided by the National Center for Health Statistics \cite{deaths_data}.}, we are instead going to have a look at which pre-existing conditions seem to have the greatest impact on COVID-19 mortality (Figure \ref{d_cond}). 

\begin{figure}[ht!]%
    \centering
    \includegraphics[width=0.85\linewidth]{latex/images/deaths_conditions.pdf}
    % \includegraphics[width=13cm]{latex/images/vacc.PNG}%
    \vspace{-0.2cm}
    \caption{Conditions contributing to COVID-19 Deaths}
    \label{d_cond}
\end{figure}

As can be seen from Figure \ref{d_cond}, influenza and pneumonia seem to be one of the main causes of deaths related to COVID-19. Taking a greater look at the patients ages whom died having these pre-conditions, we can then see how having a greater age can increase the overall likelihood of dying because of COVID-19 (Figure \ref{d_top}).

\begin{figure}[ht!]%
    \centering
    \includegraphics[width=0.65\linewidth]{latex/images/deaths_top.pdf}
    % \includegraphics[width=13cm]{latex/images/vacc.PNG}%
    \vspace{-0.2cm}
    \caption{COVID-19 Deaths Distribution}
    \label{d_top}
\end{figure}

\subsection{Survival Analysis}
\label{surv_app}
Survival Analysis is one of the most common mathematical modelling approaches which can be used in order to estimate the time it can take for a particular event to take place. This type of approach can therefore be used in order to answer causal questions such as: What is the likelihood a COVID-19 patient would die within a given time frame? How would that change given the individual specifics (e.g. pre-existing conditions) and provided cures?

The time for an event to happen (e.g. a patient death) can be characterised in this case by a continuous non-negative random variable. This random variable can then be summarised in terms of its probability density function ($g(t)$) and cumulative density function ($G(t)$). The cumulative density function at a specific time would then give us the probability that someone might have died by then (Equation \ref{cdf}).

\useshortskip
\begin{align}
\ G(t) =\int_{t}^{0} g(t) dx
\label{cdf}
\end{align}
\useshortskip

The probability that a death might have not occurred by a specific point (Survival Function), could then be specified as shown in Equation \ref{surv} as $S(t) = 1 - G(t)$.

\useshortskip
\begin{align}
\ S(t) =\int_{\infty}^{t} g(t) dx
\label{surv}
\end{align}
\useshortskip

Finally, making use of the Survival Function, we can then estimate the rate at which the different patients die in our population \footnote{Given the provided individual characteristics.} (Hazard Function). The Hazard Function, can then be considered to be as our measure of risk to die in the provided time interval (Equation \ref{hfun}).

\useshortskip
\begin{align}
\ H(t) = \dfrac{\dfrac{S(t) - S(t - dt)}{dt}}{S(t)} = \dfrac{g(t)}{S(t)}
\label{hfun}
\end{align}
\useshortskip

Making use of these basis, we can then implement a non-parametric model such as the \textbf{Kaplan-Meier Estimate} in order to create a probabilistic survival curve of the patients survival against time \cite{survival_an}.

The Kaplan-Meier Estimate can be calculated using the expression shown below. In Equation \ref{km}, $n_{i}$ represents the patients at risk at time $t_{i}$, while $d_{i}$ the number of deaths occurred so far in time.

\useshortskip
\begin{align}
\ \widehat{S(t)} = \prod_{i=t_{i}}^{t} \dfrac{n_{i}-d_{i}}{n_{i}}
\label{km}
\end{align}
\useshortskip

As a practical example, let us consider we are trying to test the efficacy of an antidote for COVID-19 in order to reduce the mortality rate of patients which suffer at the same time of hypertensive diseases. We run an experiment with 400 patients and divide them into a Control (No Antidote) group and a Treatment (Using Antidote) group of 200 patients each. Using the Kaplan-Meier Estimate, we can then be able to estimate if the antidote can have a positive impact or not. For example, in Figure \ref{s_curve} there is shown a possible outcome in case the antidote has a positive effect compared to no intervention.

\begin{figure}[ht!]%
    \centering
    \includegraphics[width=0.55\linewidth]{latex/images/survival.pdf}
    % \includegraphics[width=13cm]{latex/images/vacc.PNG}%
    \vspace{-0.2cm}
    \caption{Survival Curve}
    \label{s_curve}
\end{figure}

As shown in Figure \ref{s_curve}, all patients start with a probability to survive equal to one (before being infected) and then using the antidote can manage to limit the negative impact the disease can have on their survival probability. In Figure \ref{s_curve}, there have additionally been included confidence intervals in order to take into account uncertainties due to the reduced sample size and possibility of bias in the created data.

Another possible approach which can be used to visualise how using an antidote can be of help or not to prolong the life span of the considered patients is to plot the number of days a patient survived on a time line (Figure \ref{span}). In order to make the data visualization easier, just a random sample of 20 patients has been considered in Figure \ref{span}. Patients which have survived more than 9 days have automatically been considered as recovered in this example.

\begin{figure}[ht!]%
    \centering
    \includegraphics[width=0.5\linewidth]{latex/images/survival_span.pdf}
    \includegraphics[width=0.49\linewidth]{latex/images/survival_span2.pdf}
    % \includegraphics[width=13cm]{latex/images/vacc.PNG}%
    \vspace{-0.2cm}
    \caption{Patients survival span throughout experiment}
    \label{span}
\end{figure}

In order to make our experiment more accurate and personalised for each individual patient, we could then try to make use of additional information about the patients (e.g. age, sex, economic status). This could be easily done, making use of more advanced models such as the Cox Proportional Hazard Model. This approach has in fact been taken in different research publications such as "Risk factors for severity and mortality in adult COVID-19 inpatients in Wuhan" \cite{surv1} and "Survival Analysis of COVID-19 on Democracy with Cox Proportional Hazards Model" \cite{surv2}, so that to test new approaches to reduce the mortality rate of COVID-19. 