\chapter{Introduction}
\section{Motivation}
Autism Spectrum Disorder (ASD) is a type of neuro-developmental disorder. These types of disorders are mainly considered as disruption of regular brain functioning. Over the last few years there has been reported an increased number of children developing this pathology. Data drawn from a 2014 survey in the United States has shown an increase of 16 percent compared to the previous same survey conducted in 2012. According to this research, about 1 in 59 children in the United States have Autism Spectrum Disorder ("CDC's Autism and Developmental Disabilities Monitoring (ADDM) Network", \cite{first}). Recent research have shown that intensive early start intervention approaches can lead to major improvements to this condition. Although early intervention can be used with all the children on the ASD spectrum, response to these treatments can vary substantially from child to child. There can be many different reasons to cause diversity in improvements such as: symptom severity, joint attention, adaptive skills, interest in objects, imitation and play skills \cite{paper1}. 
\section{Objectives}
Current strategies used to alleviate autism conditions relies on three interdependent steps: 
\begin{enumerate} 
\itemsep0em
\item Early detection.
\item Approximately 25 hours per week of intensive personalised intervention. 
\item An effective monitoring of a child's progress.
\end{enumerate}
This project aims, firstly to outline the main behavioural intervention techniques used nowadays such as the The Early Start Denver Model (ESDM) and Applied Behaviour Analysis (ABA) and to then offer a suite of multi-player games as personalised delivery intervention for the children. 

These type of games are designed to be accessible both in a clinical and a home environment to facilitate intervention. In order to record the brain activity of children while playing the games, EEG cap readings are synchronised with the game and results are stored in a CSV file. These recordings will play a crucial role in observing if the game is stimulating the child and which areas are stimulated most in particular. Using this information and keeping track of the child's game scores, it will then be possible to examine what kind of improvements have been made during a predetermined time period. Analysing these results will also serve as feedback to understand what type of games can lead to most substantial improvements over others. 

Finally, the use of Machine Learning can facilitate early detection of disabilities, making use of available data to find patterns to distinguish between children suffering the disorder and those not.