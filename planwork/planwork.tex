\chapter{Conclusion}

\label{ch:planwork}
\setlength\lineskip{0pt}
\vspace*{15pt}

Overall this project had a positive outcome and all the objectives in the Project Brief have been established.

\section{Further Advancements}

Further developments can be envisaged especially concerning the Artificial Intelligence aspect of this project. The use of Wavelet analysis \cite{wavelet}, Fast Fourier Transform \cite{fft}, Entropy Analysis \cite{entrophy} or of an Auto-Encoder Network \cite{autoencoder} during the pre-processing stage could potentially reduce the noise embedded in the registered signals therefore making classification easier. Additionally, it would also be interesting to convert the registered EEG signals from the time domain to the frequency domain to see if that could improve the neural networks performances.

Another issue to be taken into account could be the reliability of the data used (eg. if the data was collected correctly, if all the noise registered during the measurements had been correctly filtered out...) and how reproducible these accuracy results can be using other forms of data which have been collected using a different types of EEG Cap or under different laboratory conditions.

When trying to use Artificial Intelligence in medical applications, high accuracy and reliability must be obtained. Explainable AI (creating models able to explain himself their classification decision making process) could help doctors to understand how/if to use AI when making decisions.